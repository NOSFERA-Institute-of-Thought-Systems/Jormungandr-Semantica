\documentclass[11pt, letterpaper]{article}

% --- PACKAGES ---
\usepackage[utf8]{inputenc}
\usepackage{amsmath, amssymb, amsthm}
\usepackage{graphicx}
\usepackage[margin=1in]{geometry}
\usepackage{hyperref}
\usepackage{algorithm, algpseudocode}
\usepackage{enumitem}
\usepackage{times}
\usepackage{listings}
\usepackage{xcolor}

% --- STYLING ---
\hypersetup{
    colorlinks=true,
    linkcolor=black,
    citecolor=blue,
    urlcolor=blue,
}
\linespread{1.1}
\lstset{
    backgroundcolor=\color{black!5},
    basicstyle=\ttfamily\small,
    breaklines=true,
    frame=single,
    rulecolor=\color{black!20},
    showstringspaces=false,
    language=Python,
    keywordstyle=\color{blue},
    commentstyle=\color{green!40!black},
    stringstyle=\color{purple},
}

% --- TITLE ---
\title{\textbf{Technical Report 1: Analysis of a Computational Bottleneck in a Scalable, Learnable Graph Operator}}
\author{Mohana Rangan Desigan}
\date{\today}

% --- DOCUMENT START ---
\begin{document}
\maketitle

\begin{abstract}
This report documents the current status and primary technical challenge of the Jörmungandr-Semantica research program. The program's goal is to develop a learnable, anisotropic graph diffusion operator for unsupervised representation learning. We chronicle the systematic failure of several hand-crafted heuristics (based on curvature and community structure), which led to the development of a scalable, differentiable learning framework built in PyTorch. This framework replaces the intractable $O(N^3)$ differentiable eigendecomposition with a scalable $O(m|E|D)$ Chebyshev polynomial approximation. While this new architecture is theoretically sound and has resolved all prior bugs, its practical implementation on CPU has revealed a significant performance bottleneck. A single training epoch on the 11,314-node 20 Newsgroups graph takes several minutes, making iterative research infeasible. This report formalizes the architecture, pinpoints the source of the bottleneck, and frames the immediate research objective: to overcome this performance barrier via GPU acceleration.
\end{abstract}

\section{Architectural Overview}
The current system is a modular pipeline designed to learn an optimal anisotropic diffusion operator.

\subsection{Core Components}
\begin{enumerate}
    \item \textbf{Graph Construction:} A k-NN graph $G=(V, E, W)$ is built from input embeddings using Faiss.
    \item \textbf{Geometric Signal:} Forman-Ricci curvature, $\kappa_{ij}$, is computed for each edge.
    \item \textbf{Learnable Operator:} A PyTorch \texttt{nn.Module}, the \texttt{DifferentiableChebyshevOperator}.
    \item \textbf{Training Loop:} Optimizes the operator's parameters $\theta$ by minimizing a triplet margin loss.
\end{enumerate}

\subsection{The Scalable, Differentiable Forward Pass}
The core of the system computes the action of the wavelet transform $X' = e^{-tL_{\text{aniso}, \theta}}X$ via a Chebyshev polynomial approximation. The key computational step is a recurrence loop:
\begin{lstlisting}
# Inside the Chebyshev recurrence loop:
Tk_X = 2 * torch.sparse.mm(L_rescaled, T1_X) - T0_X
\end{lstlisting}

\section{The Performance Bottleneck: Diagnosis}

\subsection{Problem Statement}
Despite being theoretically scalable, the practical performance on CPU is prohibitive. A single training epoch on the 20 Newsgroups dataset ($N \approx 11k, |E| \approx 118k, D=384$) takes several minutes.

\subsection{Analysis}
The system appears to hang before the first epoch's progress is logged. The dominant computation is the Chebyshev recurrence, which involves $m=30$ successive \texttt{torch.sparse.mm} operations. The bottleneck is attributed to a combination of factors:
\begin{itemize}
    \item \textbf{Computational Cost:} Each epoch requires $\approx 5.4 \times 10^9$ floating point operations.
    \item \textbf{Memory Traffic:} Intermediate products are large, dense matrices ($(11314, 384) \approx 17.4$ MB each), leading to significant latency when repeatedly accessed in CPU RAM.
    \item \textbf{Framework Overhead:} PyTorch's CPU backend for sparse operations and its associated autograd machinery are less optimized than their dense or GPU (CUDA) counterparts.
\end{itemize}

\section{Current Status and Path Forward}

We are at a critical juncture where the project is a theoretical success but a practical bottleneck.
\begin{itemize}
    \item \textbf{What Works:} The pipeline is architecturally sound. All previously identified bugs have been resolved.
    \item \textbf{What is Broken:} The performance on CPU is too slow for effective research.
\end{itemize}
The immediate research objective is to \textbf{overcome this computational bottleneck}.

% --- THE FIX IS HERE ---
% I have replaced the incorrect list formatting with a proper itemize environment
% and ensured all \textbf commands are correctly terminated.
\begin{enumerate}
    \item \textbf{Primary Path: GPU Acceleration (Kaggle/Colab).} The most direct solution is to execute the existing PyTorch code on a platform with an NVIDIA GPU and a CUDA environment. The \texttt{torch.sparse.mm} operation is highly optimized for CUDA. This is expected to yield a 10-100x speedup. This will be the focus of the next experimental sprint.

    \item \textbf{Secondary Path (Investigation): MPS Backend on Apple Silicon.} Our attempt to use the \texttt{mps} backend failed due to incomplete support for sparse tensors in PyTorch. We will continue to monitor the state of PyTorch MPS support.

    \item \textbf{Tertiary Path (Investigation): JAX Re-implementation.} A mature JAX implementation remains a viable alternative, as its JIT compiler (\texttt{@jax.jit}) is exceptionally effective at optimizing sequential operations like the Chebyshev recurrence on both CPU and GPU.
\end{enumerate}

This report concludes with the system in a ``ready for acceleration'' state. The next step is to execute the capstone experiment on a GPU-enabled platform to finally validate the performance of the learned anisotropic operator.

\end{document}
